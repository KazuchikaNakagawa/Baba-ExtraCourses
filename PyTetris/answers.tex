\documentclass[12pt, a4paper, dvipdfmx]{jarticle}
\usepackage{listings,jlisting}
\usepackage[dvipsnames]{xcolor}
\lstset{
basicstyle={\ttfamily},
identifierstyle={\small},
commentstyle={\color{OliveGreen}},
keywordstyle={\small\bfseries\color{RedViolet}},
ndkeywordstyle={\small},
stringstyle={\small\ttfamily},
frame={tb},
breaklines=true,
columns=[l]{fullflexible},
numbers=left,
xrightmargin=0zw,
xleftmargin=3zw,
numberstyle={\scriptsize},
stepnumber=1,
numbersep=1zw,
lineskip=-0.5ex
}
\title{PyTetris 解答集}
\author{Swimmy高田馬場校 有志}
\date{2024}
\begin{document}
\maketitle

\section{chapter6}
\subsection{6.2}
ブロックを表示/移動するところまでのコードです。
\lstinputlisting[caption={main.py}, language=Python]{hidden_source/chapter6/6_2/main.py}
\lstinputlisting[caption={tetris.py}, language=Python]{hidden_source/chapter6/6_2/tetris.py}

\subsection{6.3}
ブロックの動きを制限するところまでのコードです。
\lstinputlisting[caption={main.py}, language=Python]{hidden_source/chapter6/6_3/main.py}
\lstinputlisting[caption={tetris.py}, language=Python]{hidden_source/chapter6/6_3/tetris.py}
\end{document}